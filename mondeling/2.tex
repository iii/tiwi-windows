\chapter{attributeSchema objecten}

\section{Bespreek het doel en de werking van attributeSchema objecten. Hoe
kunnen deze objecten het best geraadpleegd en gewijzigd worden?}

Het AD schema is de formele definitie van alle objecten en kenmerkgegevens die
kunnen opgeslagen worden in de directory. Het is een set regels waarmee de
klassen van objecten en kenmerken in de directory, de beperkingen en limieten op
exemplaren van deze objecten en de notatie van de namen van de objecten
gedefinieerd worden. Deze definities zelf worden als objecten opgeslagen in de
schema container, ze kunnen zo op eenzelfde manier beheerd worden als alle AD
objecten. Er zijn twee types definities:
\begin{description}
	\item[Kenmerken] worden apart van klassen gedefinieerd.
	\item[Klassen] beschrijven de directory objecten die gemaakt kunnen
		worden. Elke klasse heeft een verzameling kenmerken.
\end{description}

\subsection{Doel en werking}

Een attributeSchema object is een object waarmee een kenmerk wordt ingesteld en
waarmee beperkingen opgelegd worden aan objecten die een exemplaar zijn van de
klasse met dit kenmerk. Elk kenmerk moet zo maar eenmaal gedefinieerd worden,
maar kan toch in meerdere klassen gebruikt worden.

\subsection{Raadplegen en wijzigen}

Voor ontwikkelings- en testdoeleinden kun je het AD schema bekijken en wijzigen
met adsiedit.msc, of met een specifiek hiervoor ontwikkelde snap-in, Active
Directory Schema. Er is geen standaard MMC console die deze snap-in bevat. Het
wordt geimplementeerd door schmmgmt.dll, welke na installatie nog moet
geregistreerd worden met regsvr32. De Schema snap-in laat toe, op een meer
eenvoudige wijze dan adsiedit.msc, om zowel kenmerken als klassen te bekijken,
te wijzigen en te creëeren. AD ondersteunt geen verwijdering van schemaobjecten.
Deactiveren is wel mogelijk, toch voor eigen ontwikkelde schemaobjecten, niet
voor de standaard meegeleverde objecten.

\section{Bespreek de diverse naamgevingen van attributeSchema objecten}

Alle attributeSchema objecten in het schema hebben een viervoudige naamgeving,
die alle vier uniek en gestandaardiseerd zijn:
\begin{description}
	\item[common name (cn)] is de RDN van het attributeSchema object in de
		Schema container. Opgeslagen in een kenmerk van het
		attributeSchema object met als LDAP display naam \begrip{cn}.
	\item[GUID] van het kenmerk is onafhankelijk van de GUID van het
		attributeSchema object en kan automatisch gegenereerd worden bij
		creatie van een nieuw kenmerk. Hetzelfde kenmerk zal dan wel een
		ander GUID hebben in verschillende forests. Om dit te vermijden
		kan best op voorhand een GUID gegenereerd worden. Opgeslagen in
		een kenmerk van het attributeSchema object met als LDAP display
		naam \begrip{schemaIDGUID}.
	\item[LDAP display name] is belangrijk voor programmatische toegang.
		Deze kan dikwijls, maar niet altijd, uit de common name afgeleid
		worden door alle streepjes te verwijderen, en de eerste letter
		niet in hoofdletters te vermelden. Opgeslagen in een kenmerk van
		het attributeSchema object met als LDAP display naam
		\begrip{IDAPDisplayName}.
	\item[Object identifier (OID)] die geldt als interne representatie.
		X.500 IDs worden verleend door speciale autoriteiten, en zijn
		gegarandeerd uniek in alle netwerken wereldwijd. Het is een
		decimale reeks met punten, en worden hiërarchisch toegekend. Je
		kan een OID tak aanvragen bij de regionale ISO vertegenwoordiger
		of je kan een uniek OID laten genereren in een microsoft subtak
		met oidgen. Opgeslagen in een kenmerk van het attributeSchema
		object met als LDAP display naam \begrip{attributeID}.
\end{description}

\section{Bespreek de belangrijste kenmerken van attributeSchema objecten, en hoe
die ingesteld kunnen worden.}

\begin{description}
	\item[attributeSyntax en oMSyntax] De syntax van het kenmerk bepaalt het
		data type, en zo het soort gegevens dat het kenmerk kan
		bevatten. Er zijn 26 mogelijkheden waarvan er maar 18 momenteel
		gebruikt worden in AD. Het is niet mogelijk om een nieuwe syntax
		te definiëren. Omdat men in AD bepaalde noodzakelijke data types
		niet van elkaar kan onderscheiden op basis van louter de X.500
		syntax is er een aanvullende integer waarde voorzien: OMSyntax.
	\item[rangeLower en rangeUpper] Bepalen lengte- of bereikbeperkingen van
		kenmerken.
	\item[isSingleValued] Kenmerken kunnen, afhankelijk van dit kenmerk, één
		waarde of meerdere niet-geordende waarden hebben. bv objectClass
		bevat de specifieke klasse van het object en de opeenvolgende
		klassen waarvan de klasse is afgeleid.
	\item[searchFlags] Dit bevat binaire informatie, waarbij de meeste bits
		bepalen of het kenmerk op een of andere manier geïndexeerd
		wordt. Als je een kenmerk indexeert, kun je in Active Directory
		sneller objecten zoeken die dat kenmerk hebben. Alle exemplaren
		van het kenmerk worden dan toegevoegd aan de index, niet alleen
		de exemplaren die lid zijn van een bepaalde klasse.
		\begin{itemize}
			\item De laagste bit wordt meestal gezet en bepaalt
				eenvoudige indexering van de waarde van het
				kenmerk.
			\item De tweede laagste bit zorgt voor een containerized
				index en zorgt dat objecten snel kunnen gevonden
				worden binnen een specifieke container. De
				waarde van het kenmerk wordt hiervoor
				gecombineerd met de identificatie van de
				container.
			\item Het zetten van de derde laagste bit laat ambiguous
				name resolution (ANR) toe. Bij opzoekingen waar
				men zoek gaat naar objecten waarbij minstens één
				kenmerk uit een verzameling kenmerken een
				specifieke waarde aanneemt. Bv voor displayName,
				givenName \& name staat de ANR bit op 1.
			\item Instellen van de zesde laagste bit versnelt
				opzoekingen waarin kenmerken met jokertekens
				vermeld worden.
		\end{itemize}
		De vijfde laagste bit heeft niks met indexering te maken en
		bepaalt of de waarde van attribuut behouden blijft bij het
		kopieren van het object.
	\item[systemFlags] is een binair informatieveld. De laagste bit bepaalt
		of het kenmerk gerepliceerd wordt naar andere domeincontrollers.
		Niet-gerepliceerde kenmerken worden dikwijls gebruikt om lokale
		caches te implementeren. Wordt ook gebruikt voor relatief
		dynamische kenmerken waarvan de waarde frequent wijzigt, zoals
		lastLogon en LastLogoff. LastLogonTimestamp wordt wel
		gerepliceerd.
		De derde laagste bit wijst op een geconstrueerd attribuut. Een
		geconstrueerd attribuut wordt niet opgeslagen in AD, maar wordt
		telkens opnieuw berekend. Bv canonicalName en parentGUID.
	\item[isMemberOfPartialAttributeSet] bepaalt of het kenmerk in de global
		catalog opgenomen wordt of niet.
	\item[linkID] Sommige kenmerken vormen koppels bestaande uit
		forward-link en back-link kenmerken. Indien de waarde van het
		forward-link kenmerk van een object verwijst naar de DN van een
		ander object, dan wordt het back-link kenmerk van dat object
		automatisch in- of aangevuld met de DN van het eerste object, en
		vice versa. Gebruikers met voldoende machtigingen kunnen enkel
		de waarde van forwart-link kenmerken rechtstreeks wijzigen.
		Back-link kenmerken vallen volledig onder het beheer van de
		security accounts manager component van windows server. De
		forward-link kenmerken hebben een even waarde, de backward-link
		kenmerken hebben een onevenwaarde, 1 hoger dan de forward-link
		waarde.
\end{description}

\section{Welke andere types objecten bevat het Active Directory schema, en wat
is hun bedoeling?}

\begin{description}
	\item[classSchema objecten] Net zoals voor kenmerken bevindt zich voor
		alle klassen een classSchema object in het schema. De kenmerken
		van classSchema objecten definieren de klasse, en bevatten twee
		soorten regels: structuurregels definieren de hiërarchische
		relaties tussen hetzij klassen, hetzij tussen objecten,
		inhoudsregels kenmerken definiëren die beschikbaar zijn voro een
		exemplaar van die klasse.
	\item[Abstracte schema] Naast classSchema en attributeSchema
		objecten bevat het schema nog één ander object: een subSchema
		object, het abstracte schema genoemd. Het heeft als RDN
		Aggregate en bevat een alternatieve, compacte voorstelling van
		het gehele schema. De bedoeling is om vereenvoudigde schema
		gegevens ter beschikking te stellen aan LDAP cliënts, zonder
		zich over veel implementatie details hoeven te bekommeren.
		Gecombineerd met de Active Directory Service Interfaces (ADSI)
		biedt het abstracte schema een toegang naar het schema, die veel
		meer high-level is dan via het reële schema.
\end{description}

\section{Via welke attributen kun je de klasse van een willekeurig Active
Directory object achterhalen? Hoe moet je op zoek gaan naar alle objecten van
een bepaalde klasse? Illustreer aan de hand van relevante voorbeelden.}

\begin{description}
	\item[ObjectClass] is multi-valued en niet geïndexeerd. het bevat niet
		alleen de klasse van het object zelf, maar ook alle
		hiërarchische superklassen (uitgezonderd de statische
		hulpklassen).
	\item[objectCategory] is single-valued en is geïndexeerd. Het wordt
		echter niet noodzakelijk ingevuld met de klasse van het object,
		het bevat de meest typische vertegenwoordiger uit de verzameling
		bestaande uit de klasse zelf en alle hiërarchische superklassen.
\end{description}

\subsection{voorbeelden}

Voor het opzoeken van printers is de selectie van objecten, waarvoor de
objectCategory ingesteld is op printQueue, duidelijk de beste keuze, aangezien
dit de opzoeking toelaat om op indexering een beroep te doen.

Indien men analoog gebruikers zou willen opzoeken via objecten, waarvoor de
objectCategory ingesteld is op person, dan is dit weliswaar een performante
oplossing, maar zal dit niet alleen objecten van de klasse user opleveren, maar
ook objecten van de klasse contact. Om deze uit te sluiten, zou men kunnen
overwegen om objecten te selecteren, waarvoor user tot de objectClass behoort.
Dit heeft echter het nadeel dat de zoekopdracht nu niet alleen objecten van de
klasse user zal opleveren, maar ook objecten van de ervan afgeleide klasse
computer. De enige juiste mogelijkheid in dit geval is om op zoek te gaan naar
objecten waarvoor zowel de objectCatergory ingesteld is op person, als user tot
de objectClass behoort.
