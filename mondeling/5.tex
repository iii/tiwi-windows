\chapter{Active Directory domeinstructuren}

\section{Wat is de bedoeling van vertrouwensrelaties?}

Tussen 2 domeinen kan een vertrouwensrelatie tot stand gebracht worden, zodat de
gebruikers in het ene, trusted domein kunnen geverifieerd worden door de
domeincontroller in het andere, trusting domein. Vertrouwensrelaties worden
weergegeven met een pijl in de richting van het trusted domein. Een gebruiker
kan pas toegang krijgen tot bronnen in een ander domein als er een
vertrouwenspad is van het trusting domein naar het trusting domein. Een
vertrouwenspad is een continue rij vertrouwensrelaties tussen domeinen. Dat een
gebruiker door een trusting domeincontroller is geverifieerd, wil nog niet
automatisch zeggen dat de ggebruiker toegang heeft tot de bronnen in dat domein.
Deze toegang wordt geregeld met gebruikersrechten en machtigingen die aan de
gebruiker toegekend zijn door de domeinbeheerder van het trusting domein.

\section{Bespreek de verschillende soorten vertrouwensrelaties.}

\subsection{Automatische vertrouwensrelaties}

Windows Server maakt automatisch vertrouwensrelaties aan tussen domeinen en hun
child domeinen. Deze vertrouwensrelaties kunnen niet verbroken worden en zijn
automatisch bi-directioneel en transitief. Windows Server maakt ook automatisch
vertrouwensrelaties aan tussen de trees van eenzelfde forest: de root domeinen
van alle trees in het forest vormen transitieve vertrouwensrelaties met het
forest root domein van het forest.
Aangezien Windows Server vertrouwensrelaties bi-directioneel en transitief zijn,
heeft een domein dat nieuw aangemaakt wordt in een tree of forest, automatisch
vertrouwensrelaties met alle andere Windows Server domeinen in de tree of het
forest.
Om in NT 4 hetzelfde te bekomen, moet je als systeembeheerder zelf
vertrouwensrelaties construeren, en dan nog in twee richtingen, dit met elk
bestaand domein.

\subsection{Expliciete vertrouwensrelaties}

Transitieve vertrouwensrelaties kunnen alleen bestaan tussen Windows Server
domeinen in hetzelfde forest, tenzij de diverse forests minimaal op Windows
Server 2003 functioneel niveau staan. In dat geval kun je manueel tussen de root
domeinen van de forests bi-directionele en transitieve forest trusts
configureren, waardoor je een gederatie of ream van forests krijgt. Elk koppel
domeinen in een dergelijke realm vertrouwt elkaar wederzijds. Bij meerdere
forests moet men forest trusts configureren tussen elk koppel forests.
Realm trusts, zijn een veralgemening van forest trusts, die vertrouwenspaden
leggen tussen Windows Server 2008 domeinen en willekeurige Kerberos v5 realms.
Een realm trust kan zowel bi-directioneel als enkelvoudig, en zowel transitief
als niet-transitief gedefinieerd worden.
Forest en realm vertrouwensrelaties zijn expliciete vertrouwensrelaties: trusts
die je zelf maakt, in tegenstelling tot de vertrouwensrelaties die automatisch
gemaakt worden tijdens de creatie van het domein.
Een verkorte vertrouwensrelatie is ook een expliciete vertrouwenrelatie en
maakt het mogelijk om een vertrouwenspad tussen 2 domeinen, in grote en complexe
trees, te verkorten. Dit moet het aanmeldingsproces versnellen.
Het laatste type expliciete vertrouwensrelatie is de externe vertrouwensrelatie.
Dit is een enkelvoudige vertrouwingsrelatie waarbij één domein een ander domein
vertrouwt. Deze zijn niet-transitief. Je kan geen externe relatie instellen
tussen AD domeinen in hetzelfde forest (heeft al automatische relaties). Dit kan
wel ingesteld worden tussen:
\begin{itemize}
	\item individuele domeinen in een ander forest.
	\item met NT 4 domeinen
\end{itemize}

\section{Op welke diverse manieren kunnen vertrouwensrelaties gecreëerd en
gecontroleerd worden? Bespreek ook de optionele configuratiemogelijkheden}

Enkel de expliciete vertrouwensrelaties moeten manueel geconfigureerd worden.
Dit kan op twee verschillende manieren geconfigureerd worden. Als je een
expliciete vertrouwenrelatie wil maken, moet je beschikken over de domeinnamen
en een gebruikersaccount met machtiging om vertrouwensrelaties in beide domeinen
te maken. Elke vertrouwensrelatie krijgt een wachtwoord teogewezen, dat bekend
moet zijn bij de beheerders van beide domeinen van de vertrouwensrelatie. Na het
opzetten van de vertrouwensrelatie wordt dit wachtword niet meer gebruikt.

\subsection{Active Directory Domains en Trust snap-in}

Deze snap-in is beschikbaar in domein.msc. Het aanmaken van de
vertrouwensrelatie kan door met de rechmuisknop op een domein te klikken en
achtereenvolgens Properties en de Trusts tabpagina te selecteren, en daarna de
New Trust wizard op te starten.

\subsection{Via de command-line}

In de command prompt kan je een vertrouwensrelatie configureren met de netdom
trust opdracht. Met netdom query trust krijg je een overzicht van alle
vertrouwensrelaties en hun toestand.

\subsection{Optionele configuratiemogelijkheden}
\begin{itemize}
	\item Standaard worden alle gebruikers van het trusted domein opgenomen
		in de Authenticated Users impliciete groep van het trusting
		domein. Men kan echter ook voor selective authentication kiezen,
		waardoor dit per individuele gebruiker of gebruikersgroep
		expliciet moet ingesteld worden.
	\item Indien men gebruik maakt van SID Filtering, dan wordt enkel
		rekening gehouden met de SID opgeslagen in het objectSid
		attribuut van de objecten in het trusted domein. Indien men SID
		Filtering uitschakelt, dan verwerkt het trusting domein ook de
		SIDs opgeslagen in het sIDHistory attribuut. Malfide beheerders
		in het trusted domein kunnen langs deze weg zichzelf meer
		machtigingen en rechten toeeigenen in het trusting domein.
\end{itemize}

\section{Welke verschillen zijn er in praktijk tussen NT 4.0 en Windows Server
domeinstructuren? Bespreek de alternatieve mogelijkheden bij de conversie van
een NT 4.0 domeinstructuur naar een Windows Server omgeving.}

In de praktijk maakt men in een NT 4 domeinstructuur een onderscheid tussen
master (of account) domeinen en resource domeinen. Het verschil tussen deze
domeintypes is niet in NT 4 zelf ingebouwd, maar eerder gebaseerd op het gebruik
in de praktijk. Een master domein bevat over het algemeen gebruikersaccounts en
groepen, terwijl de servers in een resource domein bronnen aanbieden. Hierbij
zijn er doorgaans tweerichtingsvertrouwensrelaties tussen alle master domeinen
onderling, en enkelvoudige vertrouwensrelaties waarbij elk resource domein elk
master domein vertrouwt.
De upgrade van NT 4 domeinstructuur naar AD begint bovenaan in de
domeinhiërarchie en gaat het proces dan naar beneden: het master NT 4 domein is
het eerste domein waarop een upgrade meot uitgevoerd worden, gevold door een
upgrade van de resource domeinen. Je kan de organisatiestructuur van de
bestaande meerdere domeinen simuleren door corresponderende OUs in het Windows
Server domein te maken.

Wanneer je een aantal NT 4 domeinen wil samenvoegen tot één groot Windows Server
domein, dan zou het eenvoudiger zijn om de oude domeinen eerst samen te voegen,
het samengevoegde domein te upgraden naar Windows Server, en vervolgens de
oorspronkelijke domein om te zetten in OUs. Hier bestaan helaas geen
hulpprogrammas voor.

\begin{itemize}
	\item WServer zal automatisch een bi-directionele vertrouwensrelatie
		leggen tussen een domein en zijn kinddomeinen. In NT 4.0 moet
		dit manueel gebeuren
	\item WServer vertrouwensrelaties zijn transitief, in NT 4.0 is dit niet
		zo
	\item WServer maakt automatisch vertrouwensrelaties aan tussen de
		verschillende trees van eenzelfde forest
\end{itemize}
