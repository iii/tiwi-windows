\chapter{Active Directory domeinstructuren}

\section{Wat is de bedoeling van vertrouwensrelaties?}

Tussen 2 domeinen kan een vertrouwensrelatie tot stand gebracht worden, zodat de
gebruikers in het ene, trusted domein kunnen geverifieerd worden door de
domeincontroller in het andere, trusting domein.

\section{Bespreek de verschillende soorten vertrouwensrelaties.}

\subsection{Expliciete vertrouwensrelaties}

manuele configuratie is hierbij vereist

\begin{description}
	\item[Forest trust] minimaal windows Server 2003 forst functioneel
		niveau nodig
	\item[Realm trust] Veralgemening van het vorige geval, Vertrouwenspaden
		tussen Windows Server 2008 domeinen en willekeurige Kerbos v5
		realms
	\item[Verkorte vertrouwensrelaties] Aanvullende transitieve
		vertrouwensrelaties tussen domeinen in complexe trees of forests
	\item[Externe vertrouwensrelatie] Enkelvoudige vertrouwensrelatie tussen
		domeinen
\end{description}

\subsection{Automatische vertrouwensrelaties}

\begin{itemize}
	\item Tussen domein en child domein
	\item eigenschappen:
		\begin{itemize}
			\item Kunnen niet verbroken worden
			\item bi-directioneel
			\item transitief
		\end{itemize}
	\item tussen trees van hetzelfde forest
	\item opmerking: in NT4 zijn vertrouwensrelaties:
		\begin{itemize}
			\item niet bi-directioneel
			\item niet transitief
		\end{itemize}
\end{itemize}

\section{Op welke diverse manieren kunnen vertrouwensrelaties gecreëerd en
gecontroleerd worden? Bespreek ook de optionele configuratiemogelijkheden}

enkel de expliciete vertrouwensrelaties moeten manueel geconfigureerd worden.

\begin{enumerate}
	\item Active Directory Domains en Trust snap-in
		\begin{itemize}
			\item Beschikbaar via domain.msc
			\item Je moet hiervoor beschikken over een
				gebruikersaccount met machtigingen in beide
				domeinen en ook de domeinnamen van beide
				domeinen
		\end{itemize}
	\item Via de command-line
		\begin{itemize}
			\item Met netdom trust kan je nieuwe vertrouwensrelaties
				toevoegen
			\item Met netdom query trust krijg je een overzicht van
				de huidige toestand van de vertrouwensrelaties
		\end{itemize}
\end{enumerate}

\section{Welke verschillen zijn er in praktijk tussen NT 4.0 en Windows Server
domeinstructuren? Bespreek de alternatieve mogelijkheden bij de conversie van
een NT 4.0 domeinstructuur naar een Windows Server omgeving.}

\begin{itemize}
	\item WServer zal automatisch een bi-directionele vertrouwensrelatie
		leggen tussen een domein en zijn kinddomeinen. In NT 4.0 moet
		dit manueel gebeuren
	\item WServer vertrouwensrelaties zijn transitief, in NT 4.0 is dit niet
		zo
	\item WServer maakt automatisch vertrouwensrelaties aan tussen de
		verschillende trees van eenzelfde forest
\end{itemize}
