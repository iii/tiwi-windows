\chapter{classSchema objecten}

\section{Bespreek het doel en de werking van classSchema objecten}

\begin{itemize}
	\item worden ook klassen of objectklassen genoemd
	\item classSchema beschrijvan de directory objecten die gemaakt kunnen
		worden
	\item elk object in AD is een instantie van een objectklasse
\end{itemize}

\section{Hoe benadert Active Directory het mechanisme van overerving?}

\begin{itemize}
	\item Elke klasse afgeleid van een andere klasse, behalve TOP
	\item Bij overerving worden alle kenmerken overgenomen van de
		superklasse, ook alle structuur- en inhoudsregels.
	\item Overname werkt recursief
	\item Meervoudige overerving is zeer beperkt en enkel mogelijk door het
		gebruik van hulpklassen.
	\item hulpklassen zijn klassen die zelf geen objecten kunnen genereren
	\item hulpklassen kunnen zowel statisch als dynamisch gebruikt worden
\end{itemize}

\section{Bespreek de diverse naamgevingen van classSchema objecten}

is analoog als bij attributeSchema-objecten
\begin{enumerate}
	\item common name
	\item GUID
	\item LDAP display name
	\item Object ID
\end{enumerate}

\section{Bespreek de belangrijkste kenmerken van classSchema objecten, en hoe
die ingesteld kunnen worden.}

\begin{itemize}
	\item Inhoudregels definiëren kenmerken die beschikbaar zijn voor
		objecten van een klasse
		\begin{enumerate}
			\item mustContain en systemMustContain
			\item mayContain en systemMayContain
			\item rDNAttID
			\item defaultSecurityDescriptor
			\item SystemOnly
			\item isDefunct
		\end{enumerate}

	\item Structuurregels definiëren mogelijke hiërarchische relaties tussen
		klassen of objecten

		\begin{enumerate}
			\item objectClassCategory
			\item defaultObjectCategory
			\item subClassOf
			\item auciliaryClass en systemAuxiliaryClass
			\item possSuperiors en systemPossSuperiors
		\end{enumerate}
\end{itemize}

\section{Welke andere types objecten bevat het Active Directory schema, en wat
is hun bedoeling?}

\begin{itemize}
	\item attributeSchema objecten
	\item Het abstracte schema
\end{itemize}

\section{Hoe en met welke middelen kan het Active Directory schema uitgebreid
worden?}

\begin{enumerate}
	\item active Directory Schema snap-in
	\item ldifde en csvde
	\item Programma's via LDAP of ADSI interface
\end{enumerate}

\subsection{hoe wijzigingen aanbrengen}

\begin{itemize}
	\item foutieve wijzigingen kunnen het volledige domein onbruikbaar maken
	\item schemauitbreiding geldt voor het gehele forest
	\item enkel gemachtigde gebruikers kunnen deze wijzigingen doorvoeren
	\item bieden wel veel potentieel dus moeten niet compleet worden genegeerd
\end{itemize}
