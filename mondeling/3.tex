\chapter{classSchema objecten}

\section{Bespreek het doel en de werking van classSchema objecten}

Klassen zijn zelf ook objecten in het schema: voor elke klasse in het schema is
er een classSchema object waarmee de klasse ingesteld wordt. Ze beschrijven de
directory objecten die gemaakt kunnen worden.
De kenmerken van classSchema objecten definiëren de klasse, en bevatten twee
soorten regels: met structuurregels definieer je de mogelijke hiërarchische
relaties tussen hetzij klassen, hetzij tussen objecten, terwijl inhoudsregels
kenmerken definiëren die beschikbaar zijn voor een exemplaar van die klasse.
Door middel van overname kunnen van bestaande klassen nieuwe klassen aangemaakt
worden.

\section{Hoe benadert Active Directory het mechanisme van overerving?}

Door overerving kan je van bestaande klassen nieuwe klassen maken. De
onmiddelijke superklasse van een klasse wordt bepaald door het kenmerk
subClassOf van het classSchema object. Speciale superklasse Top waar alle
klassen (onrechtstreeks) van worden afgeleid. Een subklasse neemt de kenmerken
van de superklasse over, inclusief de structuurregels en de inhoudsregels.
Overname werkt recursief.
Een klasse kan enkel kenmerken overnemen van zijn onmiddelijke superklasse en
van speciaal hiervoor bestemde hulpklassen. Hulpklassen zijn klassen die zelf
geen kenmerken kunnen genereren.
De kenmerken auxiliaryClass en systemAuxiliaryClass van elk classSchema object
bevatten alle mogelijke klassen waarvan deze klasse kenmerken kan overnemen.
Kenmerken met een LDAP display naam die beginnen met system zijn in
tegenstelling tot de corresponderende kenmerken niet wijzigbaar door beheerders
van het schema.
Het gebruik van hulpklassen kan zowel statisch (met auxiliaryClass kenmerk) als
dynamisch (hulpklasse opgeven bij creatie) gebeuren.

\section{Bespreek de diverse naamgevingen van classSchema objecten}

Alle classSchema objecten in het schema hebben een viervoudige naamgeving, die
alle vier uniek en gestandaardiseerd zijn:
\begin{description}
	\item[common name] is de RDN van het classSchema object in de schema
		container. Opgeslagen in een kenmerk van het classSchema object
		met als LDAP display naam \begrip{cn}.
	\item[GUID] van de klasse is onafhankelijk van de GUID van het
		classSchema object en kan automatisch gegenereerd worden bij
		creatie van een nieuwe klasse. Opgeslagen in een kenmerk van het
		classSchema object met als LDAP display naam
		\begrip{schemaIDGUID}.
	\item[LDAP display name] is belangrijk voor programmatische toegang.
		Deze kan dikwijls, maar niet altijd, uit de common name afgeleid
		worden door alle streepjes te verwijderen, en de eerste letter
		niet in hoofdletter te vermelden. Opgeslagen in een kenmerd van
		het classSchema object met als LDAP display naam
		\begrip{IDAPDisplayName}.
	\item[Object ID] die geldt als interne representatie.
		X.500 IDs worden verleend door speciale autoriteiten, en zijn
		gegarandeerd uniek in alle netwerken wereldwijd. Het is een
		decimale reeks met punten, en worden hiërarchisch toegekend. Je
		kan een OID tak aanvragen bij de regionale ISO vertegenwoordiger
		of je kan een uniek OID laten genereren in een microsoft subtak
		met oidgen. Opgeslagen in een kenmerk van het classSchema
		object met als LDAP display naam \begrip{governsID}.
\end{description}

\section{Bespreek de belangrijkste kenmerken van classSchema objecten, en hoe
die ingesteld kunnen worden.}

De kenmerken van classSchema objecten definiëren de klasse, en bevatten twee
soorten regels: met structuurregels definieer je de mogelijke hiërarchische
relaties tussen hetzij klassen, hetzij tussen objecten, terwijl inhoudsregels
kenmerken definiëren die beschikbaar zijn voor een exemplaar van die klasse.

\subsection{Inhoudsregels}

\begin{enumerate}
	\item De mustContain en systemMustContain kenmerken bevatten de lijst
		met kenmerken die verplicht zijn in elk exemplaar van de klasse.
		Een kenmerk is verplicht van zodra het verplicht is in één van
		de hiërarchische superklassen van de klasse, ook al is het voor
		de klasse zelf als optioneel gemarkeerd.
	\item De mayContain en systemMayContain kenmerken bevatten de lijst met
		kenmerken die optioneel zijn.
	\item Het rDNAttID kenmerk bepaalt welk kenmerk van een klasse gebruikt
		wordt om de RDN van objecten te bepalen. Meestal staat dit
		kenmerk ingesteld op de waarde cn (common name).
	\item Het defaultSecurityDescriptor kenmerk bepaalt de expliciete
		machtigingen die gelden voor objecten van deze klasse. Dit kan
		een eenvoudige oplossing bieden om het beheer van specifieke
		objecten te delegeren.
	\item Als SystemOnly de waarde True heeft, kunnen de structuurregels en
		de inhoudsregels van de klasse niet gewijzigd worden.
	\item Met het isDefunct kenmerk kunnen schema objecten gedeactiveerd
		worden. Zo kunnen er geen nieuwe exemplaren van de klasse
		aangemaakt worden. AD ondersteunt geen verwijdering van
		schemaobjecten.
\end{enumerate}

\subsection{Structuurregels}
\begin{enumerate}
	\item Het objectClassCategory kenmerk heeft een integer waardie die de
		categorie van de klasse bepaalt: structurele klasse (1),
		abstracte klasse (0 of 2) of hulpklasse (3). Abstracte klassen
		zijn gelijkaardig aan structurele klassen, maar kunnen zelf geen
		objecten genereren.
	\item defaultObjectCategory
	\item Het subClassOf kenmerk bepaald de onmiddelijke superklasse van een
		klasse.
	\item De auciliaryClass en systemAuxiliaryClass kenmerken bevatten alle
		mogelijke hulpklassen waarvan deze klasse kenmerken kunnen
		overnemen.
	\item De possSuperiors en systemPossSuperiors kenmerken van elk
		classSchema object definieren de mogelijke hiërarchische
		relaties tussen objecten van een klasse. Het feit of een klasse
		containerobjecten representeert, wordt niet bepaald door een
		kenmerk van de klasse zelf: een structurele klasse kan
		containerobjecten genereren van zodra een andere structurele
		klasse ernaar verwijst in zijn systemPossSuperiors of
		PossSuperiors kenmerken. De definitie van een klasse bepaalt
		voor zijn objecten bijgevolg niet van welke klasse het objecten
		kan bevatten als containerobject, maar wel van welke klasse de
		objecten als container kunnen optreden.
\end{enumerate}

\section{Welke andere types objecten bevat het Active Directory schema, en wat
is hun bedoeling?}

\begin{description}
	\item[attributeSchema objecten]
	\item[abstracte schema]
\end{description}

\section{Hoe en met welke middelen kan het Active Directory schema uitgebreid
worden?}

\begin{enumerate}
	\item active Directory Schema snap-in
	\item ldifde
	\item Programma's via LDAP of ADSI interface
\end{enumerate}

\subsection{hoe wijzigingen aanbrengen}

\begin{itemize}
	\item foutieve wijzigingen kunnen het volledige domein onbruikbaar maken
	\item schemauitbreiding geldt voor het gehele forest
	\item enkel gemachtigde gebruikers kunnen deze wijzigingen doorvoeren
	\item bieden wel veel potentieel dus moeten niet compleet worden genegeerd
\end{itemize}
