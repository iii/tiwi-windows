\chapter{Active Directory replicatie}

\section{Wat is de bedoeling van replicatie?}

Gebruikers en services moeten op elk gewenst moment vanaf elke computer in het
forest toegang kunnen krijgen tot de directory gegevens. Een domein zonder
actieve domeincontroller functioneert niet langer naar gebruikers toe. Doordat
in één domein met meerdere domeincontrollers kan gewerkt worden, kan men de
fouttolerantie en de belastingsverdeling verbeteren. Bij aanpassingen (CRUD) aan
de AD gegevens moeten deze aanpassingen doorgegeven worden aan andere
domeincontrollers.

AD omvat een replicatieservice waarmee directory gegevens gedistribueerd worden
in het netwerk. Alle domeincontrollers in een domein nemen deel aan de
replicatie en bevatten een volledige kopie van alle directory gegevens voor het
eigen domein. Analoog beschikken alle domeincontrollers in een forest over het
directory schema en de configuratiegegevens, en wordt deze informatie
gedistribueerd doorheen het hele forest. De replicatieservice zorgt ervoor dat
directory gegevens altijd actueel zijn. Behalve voor wat de operations master
functies betreft, zijn alle Windows Server domeincontrollers in een domein dan
ook equivalent.

\section{Hoe wordt dit in Windows Server (ondermeer te opzichte van NT4)
gerealiseerd: Bespreek de verschillende technische kenmerken en concepten van
Windows Server replicatie, en hoe men specifieke problemen vermijdt en oplost.}

In AD wordt multi-master replicatie gebruikt, zodat de directory kan bijgewerkt
worden vanaf elke domeincontroller. Dit is een evolutie van het in NT4 gebruikte
master-slave model met primaire domeincontrollers en back-up controllers.

Het is belangrijk om ervoor te zorgen dat het replicatie verkeer relatief
beperkt blijft. In AD worden daarom alleen gewijzigde directory gegevens
gerepliceerd. Om ervoor te zorgen dat een specifieke wijziging niet meerdere
malen naar dezelfde domeincontroller wordt gerepliceerd, wordt er gebruik
gemaakt van een 64-bits Update Sequence Number (USN). Samen met het GUID van een
domeincontroller wordt dit de Up-to-Dateness Vector (UTD vector) genoemd. Elke
domeincontroller houdt in een tabel bij wat de meest recente UTD vector is die
hij van elke andere controller ontvangen heeft. Dit uitwisselingsprotocol is op
te volgen voor elke individuele partitie met behulp van de repadmin opdracht.

De metadata van elk object houdt van elk kenmerk ondermeer een Property Version
Number (PVN) bij, samen met de corresponderende UTD vector (van de
domeincontroller die de wijziging uitgevoerd heeft, op het ogenblik van de
wijziging). Op basis van de UTD vectortabel kan een controller alle wijzigingen
met een bepaalde minimale USN aan een andere controller opvragen. Aan de hand
van de metadata kan deze controller precies te weten komen welke wijzigingen hij
moet opsturen.

Met het multi-master model is het mogelijk dat hetzelfde kenmerk op meer dan één
domeincontroller tegelijkertijd gewijzigd wordt. Dit wordt opgelost door de
wijziging op het oudste tijdstip te negeren. Bij exact dezelfde tijd wordt de
wijziging van de domeincontroller met het hoogste GUID geaccepteerd. Voor
sommige objecten (schema objecten) is dit niet goed genoeg en kunnen verzoeken
tot wijzigingen enkel door 1 specifieke domeincontroller verwerkt worden. Bij 
schema objecten staat de schema master in voor de verwerking, waarvan er 
maximaal 1 in het forest is.

Verwijdering van objecten biedt een bijkomend probleem: men moet vermijden dat
het object opnieuw gecreëerd wordt door replicatie vanuit een andere
domeincontroller. Een object wordt hiervoor niet onmiddelijk uit AD verwijderd,
maar als tombstone object gemarkeerd, en in een hidden container (Deleted items)
geplaatst. Pas echt verwijderd na tombstoneLifeTime (default 60 of 180) dagen.

Nog een belangrijk verschil met NT4 is de \begrip{store-and-forward} replicatie:
elke verandering op een specifieke controller wordt slechts uitgewisseld met
enkele andere domeincontrollers, die op hun beurt de wijzigingen communiceren
met nog enkele andere domeincontrollers. De gebruikte replicatietopologie wordt
automatisch gegenereerd door de Knowledge Consistency Checker (KCC) software op
elke AD domeincontroller. Via de hierbij aangemaakte verbindingsobjecten zoekt
AD zelf de meest optimale manier om het repliceren zo snel mogelijk uit te
voeren. Er worden hierbij individuele topologieën gecreëerd voor de
domeingegevens enerzijds, en voor de schema en configuratiegegevens anderzijds.
Ook voor elke applicatiepartitie is er een aparte replicatietopologie.

Windows Server replicatie is een pull, en geen push mechanisme. De
domeincontrollers brengen elkaar wel op de hoogte van de wijzigingen, maar het
opvragen gebeurt op initiatief van de replicatiepartner. Wijzigingen worden ook
opgespaard in tijdsintervallen van standaard vijf minuten en pas daarna
verstuurd. Dit wordt propagation damping genoemd en wordt uitgeschakeld voor
objecten die met beveiliging te maken hebben. Automatisch om het uur: polling als
er geen meldingen van een domeincontroller zijn gekomen.

Windows Server replicatie is multi-threaded: een domeincontroller kan simultaan
repliceren met diverse partners. Een laatste verschil met NT4 is de kleinste
replicatie entiteit: in NT4 was dit een volledig object, in windows 2000
domein functioneel niveau is dit een heel attribuut en vanaf windows server 2003
functioneel niveau is de atomaire waarde van een attribuut (1 veld in
multi-valued attribuut).

\section{Welke toestellen repliceren onderling in een forest? Welke
specifieke gegevens worden hierbij uitgewisseld}

Domeincontrollers in een domein repliceren de domeingegevens van dat domein met
elkaar. Naar elke Global Catalog wordt een subset van de domeingegevens van elk
domein gerepliceerd. Het schema en configuratiegegevens worden gerepliceerd naar
alle domeincontrollers in het forest.

\section{Welke impact hebben sites met betrekking tot de replicatie van Active
Directory gegevens? Je hoeft het begrip site op zich niet verder te behandelen.}

Binnen sites is er automatisch en continu replicatie. Tussen sites onderling is 
het de bedoeling om een evenwicht te vinden tussen enerzijds behoefte aan 
actuele directory gegevens en anderzijds beperkingen die door beschikbare 
netwerk bandbreedte opgelegd worden.

Inter-site replicatie vertoont heel wat implementatie verschillen met intra-site
replicatie:
\begin{itemize}
	\item Enkel polling, urgent replication is bijgevolg niet mogelijk.
	\item Standaard compressie, kan uitgeschakeld worden, reduceert volume
		tot 40\%
	\item Met behulp van sitekoppelingen aan te geven hoe verschillende
		sites onderling, met rechtstreekse fysieke netwerkverbindingen,
		verbonden zijn.
		\begin{itemize}
			\item KCCs genereren automatisch enkel
				verbindingsobjecten tussen sites als er tussen
				beide sites een sitekoppeling bestaat.
			\item Fouttolerantie: AD houdt zoveel mogelijk rekening
				met meerdere routes tussen sites om wijzigingen
				te repliceren.
			\item Alternatieve sitekoppeling maken de replicatie
				betrouwbaarder, als backup voor een falende
				sitekoppeling/verbinding.
		\end{itemize}
\end{itemize}
