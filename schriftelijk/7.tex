\chapter{Gebruikersprofielen}

\section{Wat is de bedoeling van gebruikersprofielen? Bespreek zowel uit het
standpunt van gebruikers, als uit het standpunt van beheerders, en bespreek
hierbij de voor- en nadelen.}

Op NT4+ computers worden de desktop instellingen voor de werkomgevingen van
gebruikers op de lokale computer automatisch gedefinieerd en bijgehouden met
behulp van gebruikersprofielen. Dit is in feite een momentopname van de desktop
omgeving van een gebruiker. Het definieert een aangepaste desktop omgeving met
individuele weergave instellingen, netwerkverbindingen, printerverbindingen en
andere spcifieke instellingen. Gebruikersprofielen bieden diverse voordelen:
\begin{itemize}
	\item Elke gebruiker beschikt over persoonlijke desktop instellingen bij
		het aanmelden.
	\item Bij aanmelden worden automatisch de desktop instellingen hersteld
		die actief waren bij afmelding.
	\item Ze kunnen ook centraal op een server worden opgeslagen. We spreken
		dan van roaming gebruikersprofielen.
\end{itemize}

Gebruikersprofielen zijn afkomstig uit het NT4 tijdperk en niet strikt
noodzakelijk in NT5+ omgevingen. Group Policies kunnen namelijk de instellingen
van het gebruikersprofiel te niet doen. Maar het blijft wel de eenvoudigste
manier om gebruikersinterfaces te configureren.

Met gebruikersprofielen is het voor beheerders mogelijk om voor elke gebruiker
een individuele desktop te configureren. Ook kan er één algemeen
gebruikersprofiel aangemaakt worden, dat verplicht door alle gebruikers moet
gebruikt worden. Gebruikers kunnen terwijl ze aangemeld zijn hier wel
wijzigingen aan doen, maar deze zullen niet opgeslagen worden en bij de volgende
aanmelding zal terug het standaard profiel verschijnen.

\section{Geef de verschillende types gebruikersprofielen. Hoe worden deze
ingesteld, en waar worden deze bij voorkeur opgeslagen?}

Er zijn 3 belangrijke soorten gebruikersprofielen:
\begin{description}
	\item[Lokale gebruikersprofielen] Deze worden automatisch aangemaakt als
		een gebruiker zich voor de eerte keer aanmeld op een computer.
		Het wordt op de lokale harde schijf opgeslagen. Alle wijzigingen
		in dit profiel zijn specifiek voor de gebruiker.
	\item[Roaming gebruikersprofielen] Dit zijn gebruikersprofielen die op
		een netwerk share worden opgeslagen. Een gebruiker kan op gelijk
		welke computer in het netwerk inloggen en zijn eigen
		gebruikersprofiel zal gedownload worden van de share en getoond
		worden. Het instellen van een roaming profiel kan door het
		profilePath attribuut van een gebruikersobject in te vullen.
		Hier wordt een netwerk pad opgegeven in de vorm: \textbackslash
		\textbackslash servernaam\textbackslash sharenaam\textbackslash
		mapnaam. De mapnaam bevat waarschijnlijk een component van de
		vorm \%username\%.
		Bij het afmelden wordt een kopie van het al dan niet gewijzigde
		gebruikersprofiel opgeslagen, zowel lokaal als in de netwerkmap.
		Bij terug aanmelden wordt de timestamp van het lokale en roaming
		profiel vergeleken, waarmee kan bepaalt worden of het profiel
		moet gekopieerd worden naar de lokale computer.
		Roaming profielen bieden samengevat drie grote voordelen:
		mobiliteit, fouttolerantie en centrale beheerbaarheid. De
		nadelen zijn een verhoogd netwerkverkeer, langere aanlog- en
		uitlogtijden en een verhoogd veiligheidsrisico. NT5 en NT6+
		roaming profielen zijn helaas onderling niet uitwisselbaar. Men
		noemt ze respectievelijk v1 en v2 profielen.
	\item[Mandatory gebruikersprofielen] Deze worden gedefinieerd door een
		systeembeheerder en zijn verplicht te gebruiken voor alle
		gebruikers. Wijzigingen die de gebruikers aanbrengen worden niet
		opgeslagen. In tegenstelling tot roaming profielen, kunnen
		mandatory profielen gedeeld worden door diverse gebruikers. Ze
		worden bij voorkeur opgeslagen in de SYSVOL share van
		domeincontrollers. Mandatory profielen bieden systeembeheerders
		de meeste controle, maar eisen anderzijds ook het meeste setup
		werk.
\end{description}

\section{Geef de verschillende componenten van gebruikersprofielen.}

Een gebruikersprofielmap bevat twee belangrijke componenten:
\begin{itemize}
	\item Het Hive bestand NTuser.dat vormt het register gedeelte van het
		gebruikersprofiel. Dit omvat ondermeer alle door de gebruiker
		gedefinieerde instellingen voor de Windows omgeving en
		toepassingsprogramma's. NTuser.dat.log is een log bestand van
		handelingen dat er is om NTuser.dat te beschermen als er
		veranderingen worden weggeschreven naar schijf.
	\item Bestanden en snelkoppelingen naar verschillende desktop items,
		zoals naar toepassingsspecifieke gegevens, locatie op het
		internet, en de laatst gebruikte documenten. Bestanden en
		snelkoppelingen worden gestructureerd in diverse mappen. Om de
		grootte van het gebruikersprofiel te beperken, worden er bij
		voorkeur enkel snelkoppelingen in het profiel opgenomen, en
		wordt een deel van het profiel, appdate en local settings, enkel
		lokaal bewaard.
		Op NT6+ worden de snelkoppelingen uit het persoonlijk v2 profiel
		dynamisch aangevuld met de items in de overeenkomstige submappen
		van het lokale public profiel.
\end{itemize}

\section{Over welke alternatieve hulpmiddelen beschikt een beheerder om
gebruikersprofielen te configureren?}

Een systeembeheerder kan, door de Default user, Default user.V2, Default, All
users en Public profielen te manipuleren, standaardinstellingen definiëren die
opgenoemen worden in alle individuele gebruikerprofielen. Deze profielen in de
NETLOGON share vormen een uitstekende combinatie met roaming profielen.

Een andere optie is dat de beheerder voorgedefinieerde gebruikersprofielen
maakt. Hiervoor log je in met een willekeurig gebruikersaccount en maak je als
gebruiker het gewenste gebruikersprofiel. Daarna kopieer je als administrator
het gebruikersprofiel naar de gewenste netwerk share. Aangezien de NTuser.dat
component intern ACLs bevat op registersleutels, kan een op bestandbasis
gekopieerd gebruikersprofiel enkel gebruikt worden door de gebruiker die het
profiel heeft gecreëerd. Om dit te verhelpen zijn er twee opties:
\begin{itemize}
	\item Wijzig met de register editor de ACLs op de registersleutels in
		NTusers.dat.
	\item Gebruik een specifiek met Windows Server meegeleverd
		hulpprogramma. Hiermee kan opgeven wie het profiel mag gebruiken
		in het Permitted to use veld.
\end{itemize}
