\chapter{Gebruikersgroepen}

\section{Bespreek in detail het onderscheid tussen de diverse soorten
veiligheidsgroepen, ondermeer afhankelijk of het toestel al dan niet in een
domein is opgenomen. Behandel hierbij vooral de mogelijkheden en beperkingen.}

Bespreek ondermeer:
\begin{itemize}
	\item De zichtbaarheid van de diverse soorten groepen
	\item Welke objecten er lid van kunnen zijn
	\item De onderlinge relaties en de regels voor het nesten van de diverse
		soorten groepen? Stel deze relaties eveneens schematisch voor.
\end{itemize}

Er zijn drie soorten veiligheidsgroepen, bereiken of scopes genoemd: lokaal,
globaal en universeel. Zowel het type als de scope van een groep worden
bijgehouden in de individuele bits van het groepType attribuut van het groep
object. Het bereik van een groep bepaalt zowel of een groep leden uit andere
domeinen en forests kan hebben, als de domeinen waarin rechten en machtigingen
aan de groep kunnen toegewezen worden.

\subsection{Lokale veiligheidsgroepen}

Lokale groepen kunnen leden uit elk domein van het forest of andere trusted
domeinen bevatten. Lokale groepen zijn enkel zichtbaar en geldig in het eigen
domein. Lokale groepen worden niet gekopieerd naar de global catalog. Ze zijn
niet enkel zichtbaar voor domeincontrollers maar ook op alle werkposten en
lidservers van het domein.

Lokale groepen worden typisch gebruikt om rechten en machtigingen toe te kennen,
en bevatten eerder andere groepen dan gebruikers.

\subsection{Globale veiligheidsgroepen}

Globale groepen kunnen alleen gebruikers en andere globale groepen uit hetzelfde
domein omvatten. Ze zijn zichtbaar in elk domein van het forest of andere
trusting domeinen. Globale groepen kunnen ook toegelaten worden tot de lokale
groepen van werkposten en lidservers. De naam van de groep wordt gekopieerd naar
de global catalog, maar niet de leden ervan.

Ze worden vooral gebruikt als container voor gebruikers die dezelfde
machtigingen of rechten nodig zullen hebben.

\subsection{Universele veiligheidsgroepen}

Universele groepen zijn nieuw voor Windows Server en zijn enkel te gebruiken in
domeinen met minstens Windows 2000 native functioneel niveau. Ze kunnen leden
uit elk domein van het forest (niet van andere trusted domeinen) bevatten en
zijn zichtbaar in elk domein van het forest (niet in andere trusting domeinen).

Universele groepen worden net als lokale groepen typisch gebruikt om rechten en
machtigingen toe te kennen, maar bieden het voordeel dat ze in alle domeinen
tegelijkertijd geldig zijn. Ze zijn belastend voor de global catalog. Bij
voorkeur bevatten universele groepen dan ook enkel globale groepen.

\section{Hoe en waarom worden deze soorten groepen in de praktijk best gebruikt,
al dan niet gecombineerd? Van welke omstandigheden is dit afhankelijk?
Illustreer aan de hand van concrete voorbeelden.}

\begin{description}
	\item[Lokale groepen] worden typisch gebruikt om rechten en machtigingen
		toe te kennen, door voor elke bron of verzameling bronnen één of
		meerdere lokale groepen te creëren en de toegang tot die bron in
		te stellen door één keer toegang te verlenen aan de lokale
		groepen. De toegang tot de bron wordt nadien enkel gewijzigd
		door het lidmaatschap te manipuleren. Lokale groepen zijn niet
		alleen zichtbaar op domeincontrollers, maar ook op alle
		werkposten en lidservers. Dit neemt de noodzaak weg om uit
		veiligheidsoverwegingen alle lidservers te configureren als
		domeincontrollers.
	\item[Globale groepen] worden eerder gebruikt als containers voor
		gebruikers die dezelfde machtigingen of rechten nodig zullen
		hebben en als leden van andere groepen, in het domein zelf of in
		een trusting domein. Als je een verzameling gebruikers,
		afkomstig uit een ander domein toegang wil verlenen tot een
		bepaalde bron, dan kun je deze verzameling best groeperen in een
		globale groep (van het vreemde domein) en deze lid maken van een
		lokale groep, gekoppeld aan de bron.
	\item[Universele groepen] verenigt op het eerste zicht de beste
		karakteristieken van zowel lokale als globale groepen.
		Universele groepen worden net als lokale groepen typisch
		gebruikt om rechten en machtigingen toe te kennen, maar bieden
		het voordeel dat ze in alle domeinen tegelijkertijd geldig zijn
		en dus slechts eenmaal moeten gedefinieerd worden. Zowel de
		namen als de leden van universele groepen worden in de global
		catalog opgenomen, en dus wordt deze zeer groot. Bij voorkeur
		bevatten universele groepen enkel globale groepen.
\end{description}

\section{Welke conversieregels gelden er tussen de diverse soorten groepen.
Behandel hierbij alle mogelijke combinaties.}

Lokale groepen mogen niet naar universele groepen geconverteerd worden als die
lokale groepen andere lokale groepen bevatten. Dit omdat een universele groep
geen lokale groep kan bevatten.

Als een globale groep een andere globale groep bevat kan dat lid niet
geconverteerd worden naar een universele groep. Dit omdat globale groepen geen
universele groepen kunnen bevatten.

Er mogen verschuivingen optreden als er geen regels gebroken worden.

Universele groepen kunnen terug naar zowel globale groepen als lokale groepen
teruggebracht worden op voorwaarde dat een universele groep leden heeft die tot
1 domein behoren.

Een globale groep wordt per domein gedefinieerd in tegenstelling tot een
universele groep dat in de global catalog opgeslagen wordt en dus over de
verschillende domeinen van het forest gekend is. Dus als een universele groep
leden bevat over meerdere domeinen, is deze conversie niet mogelijk.
